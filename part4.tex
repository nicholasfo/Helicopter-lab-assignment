\section{Part IV - State estimation}
In this part we will deal with state estimation using an observer. In the prevoius problems we have used numerical differentiation, but in this task we will estimate the nonmeasured states in our system instead. 

%%%%%%%%%%%%%%%%%%%%%%%%%
\subsection{Problem 1}

Our system has the same state-space form as in the previous tasks

\begin{align}
  \begin{split}
    \dot{\bm{x}} &= \bm{Ax} + \bm{Bu} \\
    \bm{y} &= \bm{Cx}
  \end{split}
\end{align}

where \textbf{A}, \textbf{B} and \textbf{C} are matrices. The vectors $\bm{x}$, $\bm{u}$ and $\bm{y}$ are vectors controlling state, input and output

\begin{equation}
  \label{eq:state_space_vectors}
  \bm{x} =
  \begin{bmatrix}
    \tilde{p} \\
    \dot{\tilde{p}} \\
    \tilde{e} \\
    \dot{\tilde{e}} \\
    \tilde{\lambda} \\
    \dot{\tilde{\lambda}} \\
  \end{bmatrix}
  , \quad \bm{u} =
  \begin{bmatrix}
    \tilde{V_s} \\
    \tilde{V_d} \\
  \end{bmatrix} ,
  \quad \bm{y} =
  \begin{bmatrix}
    \tilde{p} \\
    \tilde{e} \\
    \tilde{\lambda}\\
  \end{bmatrix}
\end{equation}

\newline

Hence we can derive the following matrices for \textbf{A}, \textbf{B} and \textbf{C}

\begin{equation}
  \label{eq:state_space_A_B_C}
  \bm{A} =
  \begin{bmatrix}
    0 & 1 & 0 & 0 & 0 & 0 \\
    0 & 0 & 0 & 0 & 0 & 0 \\
    0 & 0 & 0 & 1 & 0 & 0 \\
    0 & 0 & 0 & 0 & 0 & 0 \\
    0 & 0 & 0 & 0 & 0 & 1 \\
    K_3 & 0 & 0 & 0 & 0 & 0 \\
  \end{bmatrix}
  , \quad \bm{B} =
  \begin{bmatrix}
    0 & 0 \\
    0 & K_1 \\
    0 & 0 \\
    K_2 & 0 \\
    0 & 0 \\
    0 & 0 \\
  \end{bmatrix}
  \quad \text{and} \quad \bm{C} =
  \begin{bmatrix}
    1 & 0 & 0 & 0 & 0 & 0 \\
    0 & 0 & 1 & 0 & 0 & 0 \\
    0 & 0 & 0 & 0 & 1 & 0 \\
  \end{bmatrix}
\end{equation}

\newline

Furthermore $K_1$, $K_2$ and $K_3$ are given from Figure (\ref{eq: k1k2k3}).
\newpage

%%%%%%%%%%%%%%%%%%%%%%%%%
\subsection{Problem 2}

To find our observability matrix with a system consisting of 6 states the following matrix has to be used:
\begin{equation}
  \bm{\mathcal{O}} =
  \begin{bmatrix}
    \bm{C} \\
    \bm{CA} \\
    \bm{C*A^2} \\
    \bm{C*A^3} \\
    \bm{C*A^4} \\
    \bm{C*A^5} \\
  \end{bmatrix} \\
\end{equation}

This computation can easily be done in MATLAB by \textit{obsv(\textbf{A},\textbf{C})}. The computation gives us a 18x6 matrix with a rank 6, which gives us full rank and hereby this system is fully observable. \newline

Our observer is written in the following form
\begin{equation}
    \bm{\dot{\hat{x}} = A\hat{x} + Bu + L(y-C\hat{x})}
\end{equation}

The eigenvalues of (\textbf{A-LC}) are the poles of our closed loop observer. Because our system is observable we are able to choose these poles as preferred. When choosing these poles one must remember to choose them in such a way that they are driving the error to zero. It is also important to remember that the poles of our observer should converge about ten times faster than the real system. Hence the observer should be faster than \textbf{(A-BK)}.

With MATLAB one can easily find the \textbf{L} matrix by using the following function 
$L = place(A^{T}, C^{T}, poles)^{T}$. \newline

In figure \ref{fig:P4p21_e_dot} to figure \ref{fig:P4p22_p} one can see plots with the controller from part III and with the implemented integral effect. 
\newpage







%%%%%%%%%%%%%%%%%%%%%%%%%
\subsection{Problem 3}

To show that the system is observable only if $\tilde{e}$ and $\tilde{\lambda}$ is measured, our $\bm{C}$ matrix can be set to the following

\begin{equation*}
  \bm{C} =
  \begin{bmatrix}
    0 & 0 & 1 & 0 & 0 & 0 \\
    0 & 0 & 0 & 0 & 1 & 0
  \end{bmatrix}
\end{equation*}

By using the function \textit{obsv(\textbf{A},\textbf{C})} in MATLAB, our observer matrix is set to be 12x6, having a rank of 6. Hence it is observable. 

On the other hand, our $\bm{C}$ matrix will look like the following when only $\tilde{p}$ and $\tilde{e}$ are measured

\begin{equation*}
  \bm{C} =
  \begin{bmatrix}
    1 & 0 & 0 & 0 & 0 & 0 \\
    0 & 0 & 1 & 0 & 0 & 0
  \end{bmatrix}
\end{equation*}

Through \textit{obsv(\textbf{A},\textbf{C})} we can generate our observer matrix, which also in this case is a 12x6 matrix. On the other hand, this observer has the rank 4, and it is clearly not observable. Plot can be seen in Figure \ref{fig:P4p3-e_dot} \newline

Despite that our system is observable, it is not necessarily controllable with our new observer. While maneuvering the helicopter it usually ends up crashing. The reason for this is because the travel state is dependant of the pitch, a state which we are not measuring. 